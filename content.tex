\begin{frame}[fragile]{Finite Automaton: Example 1}
  \begin{center}
    \begin{tikzpicture}[auto, on grid, node distance=2cm, initial text=, >=latex]
      \node[state, initial]   (q_0)                {$q_0$}; 
      \node[state, accepting] (q_1) [right of=q_0] {$q_1$};
      \node[state]            (q_2) [right of=q_1] {$q_2$}; 
      \path[->] 
        (q_0) edge [loop above] node {$0$}   ()
              edge []           node {$1$}   (q_1)
        (q_1) edge [bend left]  node {$0$}   (q_2)
              edge [loop above] node {$1$}   ()
        (q_2) edge [bend left]  node {$0,1$} (q_1);
    \end{tikzpicture}
  \end{center}
  \pause
  \begin{center}
    Try running the automaton on the following strings. \\
    $1101$, $1$, $01$, $11$, $0101010101$, $100$, $0100$, \\
    $110000$, $0101000000$, $0$, $10$, $101000$
  \end{center}
  \citeeg{Sipser page 34}
\end{frame}


\begin{frame}[fragile]{Finite Automaton: Example 2}
  \begin{center}
    \begin{tikzpicture}[auto, on grid, node distance=2cm, initial text=, >=latex]
      \node[state, initial]   (b_0)                {$b_0$}; 
      \node[state]            (b_1) [right of=b_0] {$b_1$};
      \node[state, accepting] (b_2) [right of=b_1] {$b_2$}; 
      \path[->] 
        (b_0) edge [loop below] node {$a$}   ()
              edge []           node {$b$}   (b_1)
        (b_1) edge [loop below] node {$a$}   ()
              edge []           node {$b$}   (b_2)
        (b_2) edge [loop below] node {$a,b$} ();
    \end{tikzpicture}
  \end{center}
  \pause
  \begin{center}
    Try running the automaton on the following strings. \\
    $aaaa$, $ababa$, $bababb$, $abaa$ \\
    \pause
    \vspace{8mm}
    Describe the strings that the automaton recognises.
  \end{center}
  \citeeg{Sipser chapter 1 4(a) -- Part 2}
\end{frame}

\begin{frame}[fragile]{Finite Automaton: Example 3}
  \begin{center}
    \begin{tikzpicture}[auto, on grid, node distance=2cm, initial text=, >=latex]
      \node[state, initial]   (a_0)                {$a_0$}; 
      \node[state]            (a_1) [right of=a_0] {$a_1$};
      \node[state]            (a_2) [right of=a_1] {$a_2$}; 
      \node[state, accepting] (a_3) [right of=a_2] {$a_3$};
      \path[->] 
        (a_0) edge [loop below] node {$b$}   ()
              edge []           node {$a$}   (a_1)
        (a_1) edge [loop below] node {$b$}   ()
              edge []           node {$a$}   (a_2)
        (a_2) edge [loop below] node {$b$}   ()
              edge []           node {$a$}   (a_3)
        (a_3) edge [loop below] node {$a,b$} ();
    \end{tikzpicture}
  \end{center}
  \begin{center}
    \pause
    Try running the automaton on the following strings. \\
    $aaaa$, $ababa$, $bababb$, $abaa$ \\
    \vspace{8mm}
    \pause
    Describe the strings that the automaton recognises.
  \end{center}
  \citeeg{Sipser chapter 1 4(a) -- Part 1}
\end{frame}

\begin{frame}[fragile]{Finite Automaton: Concepts}
  \begin{center}
    \begin{tikzpicture}[auto, on grid, node distance=2cm, initial text=, >=latex]
      \node[state, initial, onslide={<2,3>{gmitblue}}]   (q_0)                {$q_0$}; 
      \node[state, accepting, onslide={<2,4>{gmitblue}}] (q_1) [right of=q_0] {$q_1$};
      \node[state, onslide={<2>{gmitblue}}]              (q_2) [right of=q_1] {$q_2$}; 
      \path[->] 
        (q_0) edge [loop above, onslide={<6>{gmitblue}}] node {\only<5>{\textcolor{gmitblue}{$0$}} \only<1-4,6>{\textcolor{black}{$0$}}}   ()
              edge [,           onslide={<6>{gmitblue}}] node {\only<5>{\textcolor{gmitblue}{$1$}} \only<1-4,6>{\textcolor{black}{$1$}}}   (q_1)
        (q_1) edge [bend left,  onslide={<6>{gmitblue}}] node {\only<5>{\textcolor{gmitblue}{$0$}} \only<1-4,6>{\textcolor{black}{$0$}}}   (q_2)
              edge [loop above, onslide={<6>{gmitblue}}] node {\only<5>{\textcolor{gmitblue}{$1$}} \only<1-4,6>{\textcolor{black}{$1$}}}   ()
        (q_2) edge [bend left,  onslide={<6>{gmitblue}}] node {\only<5>{\textcolor{gmitblue}{$0,1$}} \only<1-4,6>{\textcolor{black}{$0,1$}}} (q_1);
    \end{tikzpicture}
  \end{center}

  \begin{center}
    \only<1>{\textcolor{gmitblue}{What are the essential concepts?}}
    \only<2>{\textcolor{gmitblue}{Set of states: $Q = \{ q_0, q_1, q_2 \}$}}
    \only<3>{\textcolor{gmitblue}{Initial state: $q_0 \in Q$}}
    \only<4>{\textcolor{gmitblue}{Set of final states: $F = \{ q_1 \} \subseteq Q$}}
    \only<5>{\textcolor{gmitblue}{Alphabet: $\Sigma = \{ 0, 1 \}$}}
    \only<6>{\textcolor{gmitblue}{Transition function: $\delta = \{ ((q_0, 0), q_0), ((q_0, 1), q_1), ((q_1,0), q_2), \ldots \}$}}
  \end{center}
\end{frame}


\begin{frame}[fragile]{Deterministic Finite Automaton (DFA) definition}
  A DFA is a 5-tuple $(Q,\Sigma,\delta,q_0,F)$ where
  \begin{description}
    \item[$Q$] is a finite set of \emph{states},
    \item[$\Sigma$] is a finite set called the \emph{alphabet},
    \item[$\delta$] is the \emph{transition function} ($Q \times \Sigma \rightarrow Q$),
    \item[$q_0$] is the \emph{start state} ($\in Q$), and
    \item[$F$] is the set of \emph{accept states} ($\subseteq Q$). 
  \end{description}
  \citeeg{Sipser page 35}
\end{frame}


\begin{frame}[fragile]{Example 1 definition}
  \begin{description}
    \item[$Q =$] $\{ q_0, q_1, q_2\}$
    \item[$\Sigma =$] $\{ 0, 1 \}$
    \item[$\delta =$] $\{ ((q_0,0),q_0), ((q_0,1),q_1)$, $((q_1,0),q_2), ((q_1,1),q_1)$, $((q_2,0),q_1), ((q_2,1),q_1) \}$
    \item[$q_0 =$] $q_0$
    \item[$F =$] $\{ q_1 \}$
  \end{description}
  \citeeg{Sipser page 36}
\end{frame}

\begin{frame}[fragile]{Example 2 definition}
  \begin{description}
    \item[$Q =$] $\{ b_0, b_1, b_2\}$
    \item[$\Sigma =$] $\{ a, b \}$
    \item[$\delta =$] $\{ ((b_0,a),b_0), ((b_0,b),b_1)$, $((b_1,a),b_1), ((b_1,b),b_2)$, $((b_2,a),b_2), ((b_2,b),b_2) \}$
    \item[$q_0 =$] $b_0$
    \item[$F =$] $\{ b_2 \}$
  \end{description}
  \citeeg{Sipser chapter 1 4(a) -- Part 1}
\end{frame}

\begin{frame}[fragile]{Example 3 definition}
  \begin{description}
    \item[$Q =$] $\{ a_0, a_1, a_2, a_3 \}$
    \item[$\Sigma =$] $\{ a, b \}$
    \item[$\delta =$] $\{ ((a_0,a),a_1), ((a_0,b),a_0)$, $((a_1,a),a_2), ((a_1,b),a_1)$, $((a_2,a),a_3), ((a_2,b),a_2) \}$, $((a_3,a),a_3), ((a_3,b),a_3) \}$
    \item[$q_0 =$] $a_0$
    \item[$F =$] $\{ a_3 \}$
  \end{description}
  \citeeg{Sipser chapter 1 4(a) -- Part 1}
\end{frame}



\begin{frame}{Non-determinism}
  \begin{description}
    \item[DFAs] always have exactly one state to transition to when in any given state and reading any given symbol.
    \vspace{4mm}
    \item[One arrow] emerging from each state for each symbol. (Sometimes we use one arrow for two symbols for tidiness.)
    \vspace{4mm}
    \item[Non-deterministic] finite automata can have any number of arrows for each state and symbol.
    \vspace{4mm}
    \item[Non-determinism] simplifies automata theory, and it can be shown that NFAs and DFAs recognise the same set of languages.
  \end{description}
\end{frame}


\begin{frame}[fragile]{NFA example}
  \begin{center}
    \begin{tikzpicture}[auto, on grid, node distance=2cm, initial text=, >=latex]
      \node[state, initial]   (q_1)                {$q_1$}; 
      \node[state]            (q_2) [right of=q_1] {$q_2$};
      \node[state]            (q_3) [right of=q_2] {$q_3$}; 
      \node[state, accepting] (q_4) [right of=q_3] {$q_4$};
      \path[->] 
        (q_1) edge [loop above] node {$0,1$}        ()
              edge []           node {$1$}          (q_2)
        (q_2) edge []           node {$0,\epsilon$} (q_3)
        (q_3) edge []           node {$1$}          (q_4)
        (q_4) edge [loop above] node {$0,1$}        ();
    \end{tikzpicture}
  \end{center}
  \begin{center}
    Try running the following strings on the automaton. \\
    $111101$, $00001010$, $1110$, $\epsilon$ \\
    Describe in words the strings that the automaton recognises.
  \end{center}
  \citeeg{Sipser page 48}
\end{frame}


\begin{frame}[fragile]{NFA example}
  \begin{center}
    Construct an NFA with alphabet $\{0, 1\}$ to recognise the language $\{ w| w \textrm{ ends with } 00\}$. Try to do it with only three states.
  \end{center}
  \begin{center}
    \begin{tikzpicture}[auto, on grid, node distance=2cm, initial text=]
      \node[state, initial]   (s_0)                {$s_0$};
      \node[state]            (s_1) [right of=s_0] {$s_1$};
      \node[state, accepting] (s_2) [right of=s_1] {$s_2$};
      \path[->]
        (s_0) edge [loop below] node {0,1} (s_0)
              edge []           node {0}   (s_1)
        (s_1) edge []           node {0}   (s_2);
    \end{tikzpicture}
  \end{center}
  \citeeg{Sipser Q 1.~7(a)}
\end{frame}


\begin{frame}[fragile]{Non-deterministic Finite Automaton (NFA) definition}
  An NFA is a 5-tuple $(Q,\Sigma,\delta,q_0,F)$ where
  \begin{description}
    \item[$Q$] is a finite set of \emph{states},
    \item[$\Sigma$] is a finite set called the \emph{alphabet},
    \item[$\delta$] is the \emph{transition function} ($Q \times \Sigma_{\epsilon} \rightarrow \mathcal{P}(Q)$),
    \item[$q_0$] is the \emph{start state} ($\in Q$), and
    \item[$F$] is the set of \emph{accept states} ($\subseteq Q$). 
  \end{description}
  \vspace{5mm}
  By $\Sigma_{\epsilon}$ we mean $\Sigma \cup \{ \epsilon \}$.
  e.g. When $\Sigma = \{0,1\}$, $\Sigma_{\epsilon} = \{\epsilon,0,1\}.$
  \citeeg{Sipser page 35}
\end{frame}

\begin{frame}[fragile]{Powerset example}

  Take any set, say $A = \{0,1,2\}$.
  Its powerset is the set of all its subsets, and is denoted $\mathcal{P}(A)$.


  $$
  \mathcal{P}(A) = \Big\{ \ 
                      \{ \} \  , \  \{ 0 \} \  , \  \{ 1 \} \  ,\   \{ 2 \} \  , \ 
                      \{ 0,1 \} \  , \  \{ 0,2 \} \  , \  \{ 1,2 \} \  , \ 
                      \{ 0,1,2 \} \ 
                    \Big\}
  $$


\end{frame}